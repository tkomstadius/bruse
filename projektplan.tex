\documentclass[a4paper,12pt,oneside,final]{extbook}

\usepackage[utf8]{inputenc}
\usepackage[T1]{fontenc}

\usepackage{graphicx}
\usepackage{times}
\usepackage[english,swedish]{babel}

\usepackage{geometry}

\geometry{
 margin=20mm
}

\usepackage{fancyhdr}

\usepackage{titling}
\title{Projektplan, TNM094}
\author{\textbf{Team Bruse}\\ Daniel Rönnkvist\\ Klas Eskilson\\ Ronja Grosz\\ Erik Olsson \\ Therése Komstadius}

\frenchspacing
\setlength{\parindent}{0pt}
\parskip 5pt

\usepackage{color}
\definecolor{rltred}{rgb}{.5,0,0}
\definecolor{rltgreen}{rgb}{0,.5,0}
\definecolor{rltblue}{rgb}{0,0,1}

\usepackage[pdftex,
 colorlinks=true,
 urlcolor=rltblue,       % \href{...}{...} external (URL)
 filecolor=rltgreen,     % \href{...} local file
 linkcolor=rltred,       % \ref{...} and \pageref{...}
 citecolor=rltgreen,     % \cite{...}
 pdftitle={},
 pdfauthor={},
 pdfsubject={Projektrapport, TNM094},
 pdfkeywords={},
 pdfpagemode=,
 pdfstartview=FitH,
 bookmarks=true,
 bookmarksopen=false,
 bookmarksnumbered=true
        ]{hyperref}


\begin{document}

\pagestyle{empty}
\thispagestyle{empty}

\frontmatter

\maketitle

\pagestyle{fancy}

\chapter{Sammanfattning}

Detta är en plan för utvecklingen av en webb-baserad databas för
hantering av filer på \emph{Dropbox}.

\tableofcontents

\mainmatter

\chapter{Kort beskrivning av projektet och dess mål}

Projektet går ut på att skapa en multifunktionell webb-databas som
är tillgänglig för flera olika användare. Det ska vara möjligt för
användaren att installera tjänsten lokalt på sin egen dator. Målet
med webb-databasen är att hantera olika typer av filer och kunna
hitta dem på ett lätt och smidigt sätt med hjälp av nyckelord. Dessa
nyckelord kommer vara sökningsbara och med hjälp av dessa ska olika
typer av filer kunna visas upp enkelt och sorterade efter olika
parametrar, exempelvis nyckelord, filtyp, namn eller liknande.

\section{Frågeställningar}

\begin{itemize}
\item Hur ska en webbdatabas struktureras för att sökning efter sparade
      objekt ska kunna levereras enligt en användares förväntnignar om hastighet
      och resultat?
\item Hur går det att säkerställa att en användares information som
      lagras i en webbdatabas inte är tillgänglig för någon som inte är den
      specifika användaren eller har blivit  auktoriserad av den specifika
      användaren?
\item Hur kan filer i en webbdatabas presenteras i en webbapplikation
      på ett sätt som gör de överskådliga,  hanterbara och lättilgängliga för
      en användare, i en filstruktur utan kataloger eller annan hierarki?
\item På vilket sätt kan en webbapplikations användargränssnitt
      designas för att demonstrera all funktionalitet ett system besitter
      och göra det intuitivt för en användare?
\end{itemize}

\chapter{Tidsplan}

För att snabbt få en överblick över projektet och se distribution
gällande till exempel resurser och arbetsfördelning är \emph{Gantt}-scheman
ett bra verktyg. Varje dag under projektets gång specificeras här,
och samtliga händelser under projektets gång markeras i schemat.
Projektets Gantt schema finns att se i bilaga A.

\section{Tid för planering, efterforskningar och tekniska studier}
En övergripande efterforskning görs under projektuppstarten. Under
projektets gång kommer sedan ytterligare tekniska studier ske inför
och under varje \emph{sprint}, för att ta reda på hur vissa krav ska realiseras.

\section{Sprints och deras möten}
\label{sprintomoten}
En sprint pågår i tre veckor. I början på varje sprint kommer det
hållas ett planeringsmöte på sex timmar inför kommande sprint. Där
bestämmer gruppen tillsammans vilka scenarion som ska genomföras inför
kommande sprint. Ett sprintmål kommer att formas efter dessa
scenarier som gruppen kommer jobba mot att nå. Det är önskvärt
att ha en fungerande produkt efter varje sprint. De scenarion som
är intressant flyttas sedan över från projektets produkt-backlogg
till aktuell sprint-backlogg tillsammans med de uppgifter som finns
för att utföra arbetet.

Varje arbetsdag kommer ett 15 minuter långt dagligt Scrum-möte att hållas.
De kommer att vara stående möten utan datorer för att hålla dem
fokuserade och effektiva. Där diskuteras vad gruppmedlemmarna
gjorde förra arbetsdagen och vad de ska göra den kommande dagen
för att hjälpa till att nå målet för aktuell sprint. Där utvärderas
också möjligheten att slutföra de krav som finns i nuvarande
sprint backlogg.

Som avslutning i varje sprint kommer två möten att hållas. Först
hålls en sprintgranskning, som är tidsbestämd till tre timmar, där
det arbete som precis genomförts utvärderas. Här är det även bra
att ha med kunden som kan få en demonstration av systemet som
hittills utvecklats. Här redovisas även de problem som kan ha
stötts på och hur de har blivit lösta. Mötet ger också en möjlighet
att uppdatera projektets produkt-backlogg om nya krav eller
förändringar i existerande krav dykt upp.

Till sist hålls en sprintåterblick som är ett möte på två timmar.
Här går gruppen igenom den  sprint som genomförts utifrån sina egna
arbetsinsatser, verktyg, processer och kommunikationen i gruppen.
Det framförs konstruktiv kritik och görs upp en plan för hur
arbetet kan förbättras ur denna aspekt.

\section{Milstolpar och leverabler}
Efter varje sprint skall produkten vara ett inkrement på produkten
från föregående sprint. Detta innebär att en förbättrad version av produkten
skall finnas redo för leverans efter varje sprint.

Efter den första sprinten är målet ha en så kallad \emph{minimal viable product}.
Användaren ska kunna skapa ett konto, eller logga in på ett befintligt, och
lägga till filer från dennes Dropbox-mapp samt ge filerna nyckelord. Andra
sprinten kommer till stor del handla om att göra filerna lättåtkomliga och
sökbara för användaren. Tredje sprinten kommer att fokusera på att förbättra
användagränssnittet för att gör det mer attraktivt för användaren.
Användartester ska vara genomförda innan starten för sprint fyra. Resterande
sprintar kommer att användas för att utveckla lägre prioriterade önskemål som
delning av filer.

\section{Avstämningsmöten med ledningsgruppen}
Bör hållas ett par gånger i samband med att en sprint avslutas.
Tanken är att detta kan hållas efter sprintgranskning och sprintåterblick,
så att dessa möten kan hålla fokus på det som är
viktigt för utvecklingsprocessen.

\chapter{Infrastruktur för programmering och systemutveckling}

\section{Utvecklingsplattform}
Alla i projektet utvecklar i en \emph{UNIX}-baserad miljö. Detta valdes
för att majoriteten hade vana i denna miljö och för att förenkla
samarbetet.

\section{Kravhantering}
De kravspecifikationer som finns gällande systemets funktionalitet
från en användarsynpunkt kommer från kunden. Här är det önskvärt
att kraven blir tydligt definierade och att alla menar samma sak.
Det ligger i detta skede inget fokus på hur kraven ska uppnås.
Dessa krav diskuteras sedan av utvecklingsteamet för att ta fram
kravspecifikationer, alltså de krav som krävs för att uppnå kundens
krav. Alla kravdefinitoner sparas som scenarion i projektets
produkt-backlogg. Dessa bryts ner i mindre kravspecifikationer i form av
uppgifter. Det ska vara möjligt att genomföra ett scenario över en sprint
och en uppgift på en arbetsdag. Det är viktigt att det finns mycket
detaljer för att arbetet ska bli rätt från början. Det är önskvärt
att ha ett möte med kunden efter varje sprint för att visa upp
arbetet så långt i projektet och se om några krav har förändrats.

\section{Scenario- och uppgiftshantering}
Varje krav kommer att förvandlas till ett scenario som kommer sparas
ner i backloggen. Dessa scenarion kommer sedan att brytas ner till
mindre uppgifter. För att få en bra översikt på hur
backloggen och uppgifter hanteras kommer webbtjänsten \emph{Trello} att användas.

\section{Versionshanteringssystem och rutiner}
För att hantera projektets kod kommer \emph{Git} användas tillsammans med
\emph{Github}. Git är ett etablerat verktyg för versionshantering av kod.
För att lätt kunna dela koden lagras den på Github. Versionshantering
sker genom förgreningar från huvud- och utvecklingsgrenen där all
utveckling sker i grenarna. När utvecklingen i någon förgrening är
klar ska koden granskas av minst två andra personer innan den
sammanfogas med huvudgrenen. Koden får endast sammanfogas med
huvudgrenen av en person för att minska risken för att trasig kod.

\section{Testningsprinciper och rutiner}
Inför varje sprint kommer den testansvarige att säkerställa att det
finns tester som kan möta sprintens krav. Testen kommer att skrivas
för att upptäcka fel i den kod som skrivs för kraven, även för att
upptäcka fel som skapas. Detta kommer att ske löpande under varje
sprint efter att en funktion har implementerats.

I projektets början kommer enhetstester att implementeras. Senare
i projektet när ett användargränssnitt har byggts kommer
integrationstester att tillämpas genom användartester. Användartesterna
utförs genom att filma en testperson och skärmaktiviteten samtidigt
medan systemet manövreras.

\section{Dokumentationsprinciper och rutiner}
All dokumentation i form av textdokument och filer samlas i \emph{Google
drive}. Vid varje möte förs ett kort mötesprotokoll och alla
tavelanteckningar sparas som bilder. Backlogg samt sprints dokumenteras
på Trello som kommer agera som Scrumtavla. Backloggen beskrivs genom
olika fall där det finns kategorier för fall som är antingen vedertagna,
idéer eller slutförda. Varje sprint har en egen gren som innehåller
tillhörande scenarion, uppgifter, pågående uppgifter, uppgifter
som ska godkännas och godkända uppgifter.

För att generera lättöverskådlig dokumentation från koden används
\emph{TomDoc}. Då skrivs exempelvis klass- och funktionskommentarer i koden,
som sedan genereras till en \emph{HTML}-sida.

\section{Modelleringsstandard och rutiner}
Systemet kommer att modelleras upp som UML-modeller på tavla.
Det ger hela teamet en möjlighet att vara delaktiga och få samma bild
av systemet. Inför varje ny implementation bör nya strukturer modelleras
för att säkerställa att samtliga utvecklare har en gemensam bild av hur
strukturerna skall komma att se ut.

\chapter{Organisation}

\section{Teammedlemmar och kontaktuppgifter}
\begin{itemize}
  \item Klas Eskilson, \href{mailto:klaes950@student.liu.se}{klaes950@student.liu.se}, 073-730 73 56
  \item Ronja Grosz, \href{rongr946@student.liu.se}{rongr946@student.liu.se}, 076-218 64 26
  \item Therése Komstadius, \href{theko867@student.liu.se}{theko867@student.liu.se}, 073-940 42 28
  \item Erik Olsson \href{eriol726@student.liu.se}{eriol726@student.liu.se}, 073-840 68 72
  \item Daniel Rönnkvist, \href{danro716@student.liu.se}{danro716@student.liu.se}, 076-396 74 24
\end{itemize}

\section{Ansvarsfördelning och arbetsgrupper}
  \subsection{Erik Olsson - \emph{Scrummästaren}}
  Det är Scrummästarens uppgift att se till så att Scrum-teamet
  efterföljer Scrum-teorin och få de att förstå hur den fungerar.
  Scrummästaren hjälper också produktägaren med att hantera produkt
  backloggen på rätt sätt. Scrummästaren ska även ansvar för att
  produktutvecklingen flyter på och att det råder en god arbetsro.

  \subsection{Therése Komstadius - \emph{Produktägare}}
  Dennes uppgift är att hålla kontakt med kunden och i takt med
  det hålla kraven uppdaterade. Största ansvarsområde är att se
  till att projektets produkt-backlogg är uppdaterad och organiserad
  på ett sätt som gör att projektmålen går att nås. Det är även
  viktigt att alla förstår kraven som finns i den och att den är
  synlig och tydlig för alla.

  \subsection{Ronja Grosz  - \emph{Dokumentansvarig}}
  Dokumentansvarige har hand om att dokumentationen och
  dokumentationsprinciperna av projektet följs och utförs korrekt.
  Så som att se till att mötesprotokoll alltid förs vid varje möte
  av beslutsgrundande karaktär och att samla alla relevanta dokument.

  \subsection{Daniel Rönnkvist - \emph{Testansvarig}}
  Som testansvarig har denne som uppgift att säkerställa att tester
  skrivs som kontrollerar att kraven för den aktuella sprinten uppfylls.
  Det åligger även denne att se till så att testning sker
  kontinuerligt och att tester skrivs inför och under varje sprint.
  Denne ska även jobba för att skriva tester för att upptäcka brister
  i systemet.

  \subsection{Klas Eskilson  - \emph{Kodansvarig}}
  Uppgiften för den kodansvarige är huvudsakligen ansvarig för
  kodgranskningen inom gruppen. Koden som begärs att sammanfoga med
  redan befintlig kod skall hålla en god kvalité och inte upprepa
  sig för mycket. Det är den kodansvariges ansvar att se till att
  denna granskning görs av minst två personer som inte var involverade
  i utvecklingen av det som ska sammanfogas. Till sin hjälp har den
  kodansvarige verktyg som automatiserar detta, tex \emph{Code Climate}.

  \section{Mötesprinciper och rutiner}
  Dagar börjar 08:30 och slutar 17:00 om inte annat anges. För
  principer kring Scrum-relaterade möten, se kapitlet \ref{sprintomoten}.
  Mötena kommer att ske varje onsdag till fredag under vårens första period,
  diskussion om vilka arbetsdagar som kommer vara aktuella under andra
  perioden sker senare. All information angående salsbokning och mötestider
  sker via \emph{Facebook}-gruppen. Personen som bokar sal varierar beroende på hur
  många timmar denne har att utnyttja i salsbokningsystemet.

\chapter{Teknisk beskrivning}
\section{Målplattform, behov av programvara, grundläggande system-arkitektur, standarder och APIer}
Systemet kommer att utvecklas i \emph{Ruby} med ramverket \emph{Ruby on Rails}.
Detta ramverk är byggt enligt designmönstret \emph{Model-View-Controller}, där
olika delar i systemet är nedbrutna i olika komponenter. I \emph{model} hanteras
databasen och dess innehåll. I \emph{view} lagras alla olika vyer, tex hur det
ser ut när en användare ombeds logga in eller när någon besöker startsidan.
\emph{Controller} fungerar som spindeln i nätet. Här finns logik, och denna
komponent förser view med data från model.

För databashantering används Ruby on Rails inbyggda \emph{ActiveRecord}-modul,
som bygger på \emph{object relation model}. Detta innebär att det är enkelt
att strukturera upp relationer mellan olika tabeller och inlägg. Exempelvis
kan man säga att en användare har många filer, och på så sätt låta systemet
enkelt lista en användares filer, utan att skapa avancerade \emph{SQL}-frågor.

För att användarens användarupplevelse inte ska avbrytas utav att sidan laddas
om för ofta kommer många anrop till ett eget API att ske via \emph{Javascript} och med
hjälp av \emph{AngularJS} presentera detta för användaren. Detta för att sträva efter
en snabb och effektiv upplevelse för användaren. Till exempel vid sökning så
kommer anrop ske löpande och resultaten renderas i webbläsaren.

\section{\emph{Build}-miljö, \emph{IDE}, kompilator, \emph{debug}-verktyg, profileringsverktyg}
Då Ruby on rails inte kräver en kompilator kommer valet av textredigerare
vara upp till programmeraren. Ett alternativ till en vanlig textredigerare
är att använda \emph{IDE}:n \emph{Rubymine}, som går att använda gratis som
student.

De \emph{debug}-verktyg som redan är inbyggda i rails kommer i huvudsak att
användas, men även verktyg så som; \emph{byebug}, \emph{better-errors} och
\emph{did-you-mean}.

För testning kommer verktyget \emph{minitest} att implementeras och
tredjepartstjänsten \emph{Travis CI} kommer att kopplas till systemet för
att användas som ett komplement till kodgranskning för att försäkra kodstandard.
Minitest kommer även att användas för att köra test för att mäta systemets prestanda.
För att kolla \emph{test-coverage} och \emph{code smells} osv. ska tredjepartstjänsten
Code Climate användas. Detta för att automatisera processen att hitta duplicerad
kod och på så sätt snabba på och underlätta för refaktorering.

\section{Systembegränsningar, responstid, körtid}
För att systemet inte ska uppfattas som långsamt och irriterande för användaren
är målet att en sökning får ta max tre sekunder. Efter lätta undersökningar på
hur liknande tjänster levererar sin sökning är tre sekunder en rimlig begränsning.

Programmet kommer endast att utvecklas för att kunna köras i UNIX-miljö.
\emph{Windows} prioriteras inte på grund av att ingen av de tänkta servrar där
projektet kommer att köras är en Windows server. Windows-maskiner kommer
fortfarande att kunna utnyttja tjänsten via en webbläsare.

Ansvaret för vad som laddas upp på servern ligger endast på användaren.
Detta krav måste användaren acceptera för att få ta del av webbtjänsten.
Alltså är det användarens skyldighet att läsa på om reglerna som gäller för
att använda vår webbtjänst. Olagliga filer har systemadministratörerna rätt
att ta bort.

\section{Standarder, metoder och tredjeparts-APIer}
Som standard skall GitHubs stilguide för Ruby, \emph{JavaScript} och \emph{CSS}
användas. Den finns \href{https://github.com/styleguide}{här}.

För att hantera användares filer via Dropbox kommer deras \emph{API} att användas.
Detta genom den Ruby-klient som Dropbox tillhandahåller för att enkelt uppdatera,
lista och skapa filer på användarens Dropbox-konto. På sikt skall även \emph{Google}:s
och \emph{Box}:s API för filhantering undersökas för att se om det kan användas.

\section{Systemmiljö, filer/filformat, input och output}
Dropbox kommer användas som server för att underlätta hantering av filer. Därav
kommer webbtjänsten kunna hantera de filtyper Dropbox stödjer. I dagsläget innebär
detta i princip samtliga filer som går att lagra på en dators hårddisk.
Inledningsvis kommer fokus ligga på att kunna presentera bilder, länkar, \emph{pdf}-filer,
ljud-filer samt videor. För att kunna redigera en fil måste den laddas ner från
webbtjänsten till sin dator sen är det upp till användaren att själv hitta ett
passande program för att redigera filen.

\end{document}
