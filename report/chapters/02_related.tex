\chapter{Relaterat arbete}

\section{Taggar}
För att uppnå en effektiv sökstruktur på en databas finns det två olika metoder. Den första metoden kallas professionell indexering och är hierarkisk \cite{tagging}. Den andra metoden kallas folksonomi vilket bygger på att etiketter/taggar sätts på innehållet utan någon särskild hierki\cite{tagging}. De två olika metoderna har både för och nackdelar men sammanfattat kan det beskrivas som \textit{“Indexering strävar efter precision medan taggning strävar efter hantering”} \cite{tagging}.

Fördelarna med professionell indexering är att den är mer precis jämfört med folksonmoni eftersom det går att rama in den eftersökta informationen med huvudkategorier och underkategorier. Det går alltid att ta sig fram i databasens olika kataloger och till slut hitta den sökta filen. Till skillnad från professionell indexering måste sökningen ske via nyckelord (taggar) med folksonmoni och det är inte säkert att nyckelordet leder rätt. Användare kan ha olika uppfattning om vad basnivån är på innehållet vilket gör att antingen onödiga taggar läggs till eller att det finns brist på taggar. Om till exempel en Javascript-fil ska taggas kan det vara onödigt för en användare att tagga filen med “webb” medan det är nödvändigt för en annan. Därför kan ett traditionellt klassificerings system ge ett bättre resultat eftersom det är helt opersonligt.

Trots att det inte finns någon stavningskontroll när taggen skrivs in kan detta vara en fördel eftersom användaren då kan använda sig av slanguttryck och dialekter som kan precisera innehållet. Fördelen med att låta användaren stava taggarna precis som denne vill gör det lättare för personer med samma intresse att söka rätt på innehåll. Det är också mindre kostsamt att använda sig av folksonomi till skillnad från professionell indexering. Professionell indexering kräver att det finns en noga genomtänkt katalogstruktur och regler för vart innhållet ska placeras. I ett taggningssystem krävs det inte alls lika mycket eftertanke från användaren gällande vilken katalog denne ska börja leta i. Det krävs bara att söka efter ett någorlunda passande nyckelord.

\section{Gränssnitt}
För att designa ett användavänligt gränssnitt bör Normans principer\cite{norman} tillämpas för att åstadkomma en viss användarupplevelse. Principerna grundar sig i hur en människa uppfattar och tar in information. Dessa utgår ifrån sex punkter: synlighet, återkoppling, begränsningar, mappning, konsekvens och affordans.

\subsection{Synlighet}
Det är viktigt att utforma en design som gör att användaren känner sig trygg i gränssnittet. Användaren ska helst kunna förutspå vad som händer när denne klickar på en ikon eller en rubrik. Ikoner ska tala för sig själv, till exempel en ikon med en soptunna betyder kasta och ett förstoringsglas betyder sök. För att göra det lättare för användaren att hitta det mest relevanta på sidan \textit{white space} användas \cite{whitespace}. \textit{White space} är ett grafiskt utrymme som inte innehåller någon typ av information, det är till för att skapa mer “luft” i gränssnittet. Detta gör att användares fokus begränsas kring ett visst område och det blir lättare att navigera, till exempel omringas sökfältet av mycket \textit{white space} eftersom sökfältet är det mest relevanta på startsidan.

\subsection{Återkoppling}
Att ständigt ge återkoppling till vad som sker eller vad som har skett gör att användare känner sig tryggare i systemet. Dialogrutor, laddningssymboler och felmeddelanden är exempel på viktig återkoppling\cite{norman}. Utan en laddningssymbol får användaren känslan av att ingenting händer och kanske går tillbaka och upprepar steget vilket förstör hela processen.

\subsection{Begränsningar}
För att användaren inte ska göra oönskade saker i systemet är det en bra idé att begränsa alternativen\cite{norman}. Till exempel går det inte att skapa mappar för att användaren ska ha en bättre överblick på filerna. För att koppla på ett externt konto måste det först godkännas av användaren för att säkerställa att det inte sker av misstag. Om användaren glömt bort sitt lösenord går det alltid att återskapa det. Därför finns det ingen risk att användaren aldrig lyckas logga in igen på grund av att inloggningsuppgifterna har glömts bort.

\subsection{Mappning}
Mappning innebär att liknande innehåll samlas på samma plats för att hålla en god struktur\cite{norman}. Användare ska snabbt kunna hitta filens tillhörande funktioner. I Bruse samlas alla externa lagringtjänster vid sidan om varandra eftersom de fyller samma funktion.

\subsection{Konsekvens}
För att göra det lättare för användaren att minnas hur denne använder en funktion är det bra att ha en liknade design på hela hemsidan. Det blir lätt rörigt för användren om designen inte följer ett mönster. Att använda sig av samma teckensnitt och färgschema hjälper till att hålla sidan konsekvent\cite{norman}.

\subsection{Affordans}
Affordans betyder att ett objekt själv ska kunna beskriva vad det ska användas till. Till exempel är det tydligt att det går att hänga kläder på en klädkrok\cite{norman} ingen ytterligare beskrivning krävs. Samma sak gäller på en hemsida. Användaren ska inte behöva fundera över vad som kommer hända när denne klickar på exempelvis en papperskorg, något kommer raderas.

\section{Databaserat filsystem}
Ett databasbaserat filsystem använder sökning för att hitta filer genom metadata eller nyckelord. De flesta filsystem använder kataloglagring. Ett examensarbete på University of Twente i Nederländerna hade som uppgift att undersöka om ett databasbaserat filsystem kan ersätta filsystem med kataloglagring med avseende på användbarhet och förmåga att lära sig använda systemet \cite{twente}. Genom användartester drogs slutsatsen att ett databasbaserat filsystem presterade bättre utifrån dessa aspekter.
