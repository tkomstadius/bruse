\chapter{Slutsatser}

Här diskuteras vilka slutsatser som kan dras kring frågeställningarna för systemet. Vilka lärdomar teamet drog och vad dessa innebar.

\section{Frågeställningar}

\textbf{Hur ska en webbdatabas struktureras för att sökning efter sparade objekt ska kunna levereras enligt en användares förväntningar om hastighet och resultat?}

Genom att dela upp taggar för sig och filer för sig och skapa en relation
mellan de två som förklaras i kapitel \ref{sec:ror} kan användaren specificera
om den vill söka på taggar, filer eller båda två. Om det specificeras vilken
typ som eftersöks kommer systemet att kunna leverera ett snabbare resultat.

\textbf{Hur går det att säkerställa att en användares information som lagras i en webbdatabas inte är tillgänglig för någon som inte är den specifika användaren eller har blivit auktoriserad av den specifika användaren?}

Genom att arbeta med en säker sessionshantering på servern som gör att en
användare inte kan ändra en serverförfrågan för att modifiera data som utbyts.
För att ingen ska kunna få ut någon annan användares information måste systemet
försäkras om att rätt användare är inloggad och inte vara beroende på vilket id
som skickas med i en serverförfrågan. På så sätt kan endast de inlägg i
databasen som är kopplade till rätt användare säkert presenteras.

\textbf{Hur kan filer i en webbdatabas presenteras i en webbtjänst på ett sätt som gör de överskådliga,  hanterbara och lättillgängliga för en användare, i en filstruktur utan kataloger eller annan hierarki?}

Filer i en webbdatabas kan presenteras genom en databasbaserad filstruktur i en
webbtjänst. En databasbaserad filstruktur innebär att filerna visas vartefter
de söks efter med hjälp av taggar eller metadata. På så sätt kan filerna lätt
hittas, så länge användaren vet på ett ungefär vad den letar efter. Filer som
stämmer överens med alla sökord genom en lös sökning visas. Användaren kan
också redigera filens taggar till att passa sökningen bättre.

\textbf{På vilket sätt kan en webbapplikations användargränssnitt designas för
att demonstrera all funktionalitet ett system besitter och göra det intuitivt
för en användare?}

Det är viktigt att hålla en konsekvent design för att användaren ska känna sig
trygg i systemet. Funktioner som hanterar gränssnittets data ska vara lätta att
hitta för att underlätta användarens användning av systemet.

\section{Framtida arbete}

Den framtida visionen för systemet är att fler användare ska kunna interagera
med varandra genom att dela filer, projekt eller exempelvis \emph{mood boards}
med varandra. Detta kan implementeras genom att användaren kan redigera
synligheten för filer, så att de kan visas för andra genom att till exempel
skicka en länk som ger läs- och nedladdningsrättigheter. Att kunna göra
\emph{mood boards} måste också utvecklas, vilket kan göras med hjälp av
delningsbara taggar särskild vy.

För att öka säkerheten och kunna lagra filer på ett effektivare sätt kan den
framtida databashanteringen ske med hjälp av Mongodb. Detta kan göras genom att
byta ut all befintlig databaslogik till att ske med hjälp av Mongodb.

Fillagringstjänsten Box kan implementeras som ett alternativ till de externa
tjänsterna Google Drive och Dropbox. Detta skulle förverkligas genom att utöka
redan befintlig logik för hantering av filer från externa tjänster.

I dagsläget har systemet ett API för att låta utvecklaren skapa ett snabbt och
välfungerande gränsnitt som kan skicka och ta emot data från servern utan att
ladda om hela sidan. Denna del av systemet har utvecklats under hela projektets
gång utan någon gemensam strategi eller struktur. I vidare utveckling av
systemet bör detta göras om, så sökvägar (så kallade \emph{endpoints}) för
systemets API följer en gemensam konvention och att såväl server- som klientkod
kan struktureras på ett mer logiskt sätt.
