\chapter{Slutsatser}

\section{Frågeställningar}

\subsection{Struktur på webbdatabas}

\textbf{Hur ska en webbdatabas struktureras för att sökning efter sparade objekt
ska kunna levereras enligt en användares förväntningar om hastighet och
resultat?}

Genom att dela upp taggar för sig och filer för sig och skapa en relation mellan
de två som förklaras i sektion. Användaren kan specificera om den vill söka på
taggar, filer eller båda två. Om det specificeras vilken typ som söks efter
kommer systemet att kunna leverera ett snabbare resultat.

\subsection{Säkerhet}

\textbf{Hur går det att säkerställa att en användares information som lagras i
en webbdatabas inte är tillgänglig för någon som inte är den specifika
användaren eller har blivit  auktoriserad av den specifika användaren?}

Genom att arbeta med en säker sessionshantering på servern som gör att en
användare inte kan mixtra med en serverförfrågan för att ändra inlogg. För att
ingen ska kunna få ut någon annan användares information måste systemet
försäkras om att rätt användare är inloggad och inte vara beroende på vilket id
som skickas med i en serverförfrågan. På så sätt kan endast de inlägg i
databasen som är kopplade till rätt användar-id säkert presenteras och .

\subsection{Databasbaserad filstruktur}

\textbf{Hur kan filer i en webbdatabas presenteras i en webbapplikation på ett
sätt som gör de överskådliga,  hanterbara och lättillgängliga för en användare,
i en filstruktur utan kataloger eller annan hierarki?}

Filer i en webbdatabas kan presenteras genom en databasbaserad filstruktur i en
webbapplikation. En databasbaserad filstruktur innebär att filerna visas
vartefter de söks efter med hjälp av taggar eller metadata. På så sätt kan
filerna lätt hittas, så länge användaren vet på ett ungefär vad den letar efter.
Filer som stämmer överens med alla sökord genom en \emph{fuzzy search} visas.
Användaren kan också redigera filens taggar till att passa sökningen bättre.

\subsection{Användargränssnitt}

\textbf{På vilket sätt kan en webbapplikations användargränssnitt designas för
att demonstrera all funktionalitet ett system besitter och göra det intuitivt
för en användare?}

\section{Framtida arbete}
Den framtida visionen för systemet är att fler användare ska kunna interagera med varandra genom att dela filer, projekt eller mood boards med varandra.

För att öka säkerheten och kunna lagra filer på ett effektivare sätt kan den
framtida databashanteringen ske med hjälp av MongoDB.

\subsection{Slutgiltigt rapportförslag}
Denna rapport tar inte upp någonting om systemets användargränssnitt. Det beror
på att gränssnittet i skrivande stund ej var färdigutvecklat.
